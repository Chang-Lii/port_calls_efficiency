% Options for packages loaded elsewhere
% Options for packages loaded elsewhere
\PassOptionsToPackage{unicode}{hyperref}
\PassOptionsToPackage{hyphens}{url}
\PassOptionsToPackage{dvipsnames,svgnames,x11names}{xcolor}
%
\documentclass[
  11pt,
  letterpaper,
  DIV=11,
  numbers=noendperiod]{scrartcl}
\usepackage{xcolor}
\usepackage[margin=1in]{geometry}
\usepackage{amsmath,amssymb}
\setcounter{secnumdepth}{5}
\usepackage{iftex}
\ifPDFTeX
  \usepackage[T1]{fontenc}
  \usepackage[utf8]{inputenc}
  \usepackage{textcomp} % provide euro and other symbols
\else % if luatex or xetex
  \usepackage{unicode-math} % this also loads fontspec
  \defaultfontfeatures{Scale=MatchLowercase}
  \defaultfontfeatures[\rmfamily]{Ligatures=TeX,Scale=1}
\fi
\usepackage{lmodern}
\ifPDFTeX\else
  % xetex/luatex font selection
\fi
% Use upquote if available, for straight quotes in verbatim environments
\IfFileExists{upquote.sty}{\usepackage{upquote}}{}
\IfFileExists{microtype.sty}{% use microtype if available
  \usepackage[]{microtype}
  \UseMicrotypeSet[protrusion]{basicmath} % disable protrusion for tt fonts
}{}
\makeatletter
\@ifundefined{KOMAClassName}{% if non-KOMA class
  \IfFileExists{parskip.sty}{%
    \usepackage{parskip}
  }{% else
    \setlength{\parindent}{0pt}
    \setlength{\parskip}{6pt plus 2pt minus 1pt}}
}{% if KOMA class
  \KOMAoptions{parskip=half}}
\makeatother
% Make \paragraph and \subparagraph free-standing
\makeatletter
\ifx\paragraph\undefined\else
  \let\oldparagraph\paragraph
  \renewcommand{\paragraph}{
    \@ifstar
      \xxxParagraphStar
      \xxxParagraphNoStar
  }
  \newcommand{\xxxParagraphStar}[1]{\oldparagraph*{#1}\mbox{}}
  \newcommand{\xxxParagraphNoStar}[1]{\oldparagraph{#1}\mbox{}}
\fi
\ifx\subparagraph\undefined\else
  \let\oldsubparagraph\subparagraph
  \renewcommand{\subparagraph}{
    \@ifstar
      \xxxSubParagraphStar
      \xxxSubParagraphNoStar
  }
  \newcommand{\xxxSubParagraphStar}[1]{\oldsubparagraph*{#1}\mbox{}}
  \newcommand{\xxxSubParagraphNoStar}[1]{\oldsubparagraph{#1}\mbox{}}
\fi
\makeatother


\usepackage{longtable,booktabs,array}
\usepackage{calc} % for calculating minipage widths
% Correct order of tables after \paragraph or \subparagraph
\usepackage{etoolbox}
\makeatletter
\patchcmd\longtable{\par}{\if@noskipsec\mbox{}\fi\par}{}{}
\makeatother
% Allow footnotes in longtable head/foot
\IfFileExists{footnotehyper.sty}{\usepackage{footnotehyper}}{\usepackage{footnote}}
\makesavenoteenv{longtable}
\usepackage{graphicx}
\makeatletter
\newsavebox\pandoc@box
\newcommand*\pandocbounded[1]{% scales image to fit in text height/width
  \sbox\pandoc@box{#1}%
  \Gscale@div\@tempa{\textheight}{\dimexpr\ht\pandoc@box+\dp\pandoc@box\relax}%
  \Gscale@div\@tempb{\linewidth}{\wd\pandoc@box}%
  \ifdim\@tempb\p@<\@tempa\p@\let\@tempa\@tempb\fi% select the smaller of both
  \ifdim\@tempa\p@<\p@\scalebox{\@tempa}{\usebox\pandoc@box}%
  \else\usebox{\pandoc@box}%
  \fi%
}
% Set default figure placement to htbp
\def\fps@figure{htbp}
\makeatother





\setlength{\emergencystretch}{3em} % prevent overfull lines

\providecommand{\tightlist}{%
  \setlength{\itemsep}{0pt}\setlength{\parskip}{0pt}}



 


\KOMAoption{captions}{tableheading}
\makeatletter
\@ifpackageloaded{caption}{}{\usepackage{caption}}
\AtBeginDocument{%
\ifdefined\contentsname
  \renewcommand*\contentsname{Table of contents}
\else
  \newcommand\contentsname{Table of contents}
\fi
\ifdefined\listfigurename
  \renewcommand*\listfigurename{List of Figures}
\else
  \newcommand\listfigurename{List of Figures}
\fi
\ifdefined\listtablename
  \renewcommand*\listtablename{List of Tables}
\else
  \newcommand\listtablename{List of Tables}
\fi
\ifdefined\figurename
  \renewcommand*\figurename{Figure}
\else
  \newcommand\figurename{Figure}
\fi
\ifdefined\tablename
  \renewcommand*\tablename{Table}
\else
  \newcommand\tablename{Table}
\fi
}
\@ifpackageloaded{float}{}{\usepackage{float}}
\floatstyle{ruled}
\@ifundefined{c@chapter}{\newfloat{codelisting}{h}{lop}}{\newfloat{codelisting}{h}{lop}[chapter]}
\floatname{codelisting}{Listing}
\newcommand*\listoflistings{\listof{codelisting}{List of Listings}}
\makeatother
\makeatletter
\makeatother
\makeatletter
\@ifpackageloaded{caption}{}{\usepackage{caption}}
\@ifpackageloaded{subcaption}{}{\usepackage{subcaption}}
\makeatother
\usepackage{bookmark}
\IfFileExists{xurl.sty}{\usepackage{xurl}}{} % add URL line breaks if available
\urlstyle{same}
\hypersetup{
  pdftitle={Why Do Some Ships Stay Longer? A Cross-Country, Ship-Type Analysis of Median Port Time},
  pdfauthor={Chang Li},
  colorlinks=true,
  linkcolor={blue},
  filecolor={Maroon},
  citecolor={Blue},
  urlcolor={Blue},
  pdfcreator={LaTeX via pandoc}}


\title{Why Do Some Ships Stay Longer? A Cross-Country, Ship-Type
Analysis of Median Port Time}
\author{Chang Li}
\date{2025-12-10}
\begin{document}
\maketitle

\renewcommand*\contentsname{Table of contents}
{
\hypersetup{linkcolor=}
\setcounter{tocdepth}{3}
\tableofcontents
}

\begin{center}\rule{0.5\linewidth}{0.5pt}\end{center}

\newpage

\section{Abstract}\label{abstract}

\section{Introduction}\label{sec-intro}

Maritime transport is essential to global trade, with over 80\% of
international goods moved by sea. A key measure of efficiency is
\emph{time in port}, which is the duration of ships' stays at the port
for loading, unloading, and related work. Even small changes in port
time can accumulate over many ship visits, affecting supply chain
reliability, shipping costs, and the environment. Because of this,
policymakers, port authorities, shipping companies, and trade
researchers seek to understand the causes of differences in port times.
This paper examines how vessel characteristics and operational factors
influence port performance, focusing on the median time in port across
different countries and ship types.

Most previous studies on port efficiency have focused on factors at the
port or vessel level, such as infrastructure capacity, congestion, or
specific ship operations. There has been less attention to differences
between country- and ship-type groups, which can reveal how different
maritime segments operate across various national settings. This kind of
analysis is essential because patterns such as the average age or size
of vessels in a country or market segment often reflect long-term
infrastructure, regulations, or industry practices, rather than current
port conditions. This paper addresses this gap by exploring whether
vessel characteristics and market segments are associated with
differences in median port time for these groups. For example, if older
vessels typically have longer median port times, this may suggest that
countries or market segments with older fleets face slower port
operations.

To answer this question, the analysis uses vessel-arrival data from the
UNCTAD Port Call and Performance dataset and builds a regression
framework to measure how vessel characteristics like average age and
gross tonnage, commercial market segment, country-level indicators, and
their interactions relate to each other. Because the data are grouped by
country and ship type, the results show differences between these group
profiles, not individual vessels or ports. The analysis creates several
linear models with increasing detail, compares them using information
criteria and nested F-tests, and then chooses one for further study.
This final model estimates how median port time varies across countries
and ship types, while accounting for vessel characteristics.

The main findings indicate that vessel size, commercial market segment,
and country-level factors are linked to differences in median port time
across grouped operational categories. These results reveal structural
trends in maritime operations that could inform port planning, fleet
management, and discussions on trade efficiency. The paper is mainly
organized as follows. Section 2 details the data and variables used in
the analysis. Section 3 explains the modeling approach and inferential
framework. Section 4 presents the regression results and model
diagnostics. Section 5 explores the implications and limitations of the
findings.

\section{Data}\label{sec-data}

\subsection{Introduction of Variables}\label{introduction-of-variables}

The dataset used in this study is derived from UNCTAD's port performance
statistics, which combine Automatic Identification System (AIS)
vessel-tracking data and MarineTraffic port-mapping intelligence.
Observations are aggregated at the \textbf{country × ship type} level,
covering vessels of 1,000 GT or more. For each group, UNCTAD provides
summary measures of vessel fleet characteristics and port-time
performance.

The main variables used in this analysis are explained below:

\subsubsection{Response Variable}\label{response-variable}

The response variable, \textbf{median time in port (days)}, represents
the median number of days that vessels spend within port limits during
the reporting period. UNCTAD provides median time rather than the mean
time because the distribution of port stays is highly right-skewed. A
small subset of vessels may remain in port for extended periods, such as
for inspections or repairs, which can significantly increase the mean.
Consequently, the median offers a more robust measure of typical port
duration.

\subsubsection{Numerical Variables}\label{numerical-variables}

The variable \textbf{Age} represents the average age, in years, of
vessels calling at ports associated with each country and ship type
combination. In the raw dataset, this variable is labeled as
Average\_age\_of\_vessels\_years\_Value. Each vessel's contribution to
the average is weighted by its number of arrivals within the group, so
that vessels with more frequent visits exert greater influence on the
reported value.

Similarly, the variable \textbf{Size (GT)} indicates the average gross
tonnage of vessels in each country and ship type group, corresponding to
the raw variable Average\_size\_GT\_of\_vessels\_Value. As with Age,
this measure is weighted by the number of port calls, reflecting the
structural characteristics of vessels most commonly serving the group.

\subsubsection{Categorical Variables}\label{categorical-variables}

The variable \textbf{ShipType} shows the commercial market segment for
each observation, corresponding to the raw variable
\texttt{CommercialMarket\_Label}. The ship type categories include
liquid bulk carriers, liquefied petroleum gas (LPG) carriers, liquefied
natural gas (LNG) carriers, dry bulk carriers, dry breakbulk carriers,
roll-on/roll-off ships, container ships, passenger ships, and an
aggregate category labeled ``All ships.''

The variable \textbf{Country} specifies the economy where port calls
occurred, as recorded in the raw variable \texttt{Economy\_Label}. The
dataset includes economies such as Australia, Canada, China, Croatia,
Denmark, France, Germany, Greece, Indonesia, Italy, Japan, the Republic
of Korea, the Netherlands, Norway, the Philippines, the Russian
Federation, Singapore, Spain, Sweden, Türkiye, the United Kingdom, the
United States of America, and a global aggregate category labeled
``World.''

\subsubsection{Summary}\label{summary}

Each observation in the dataset represents a specific combination of
country and ship type. The main variables used for modeling are:

\begin{itemize}
\tightlist
\item
  \textbf{Median time in port} (response)\\
\item
  \textbf{Age} (average vessel age)\\
\item
  \textbf{Size (GT)} (average gross tonnage)\\
\item
  \textbf{ShipType} (commercial market category)\\
\item
  \textbf{Country} (economy identifier)
\end{itemize}

These variables describe the economic setting (Country), the commercial
category (ShipType), and the main features of vessel groups (Age and
Size). They provide the basis for the modeling in the next section.

\subsection{Data Cleaning}\label{data-cleaning}

Before modeling, we cleaned the data to make sure the dataset was
consistent, useful for analysis, and ready for regression:

\subsubsection{1. Removal of Non-Informative
Columns}\label{removal-of-non-informative-columns}

Some variables did not contain any useful information. This was either
because all their entries were missing or the columns were just footnote
markers or missing-value flags. Since these variables did not help with
the analysis or relate to our study goals, they were removed.

\subsubsection{2. Retention of a Unified Vessel-Size
Measure}\label{retention-of-a-unified-vessel-size-measure}

The dataset includes three vessel-size metrics: TEU, DWT, and GT. TEU is
used only for container ships, while DWT is mainly for bulk carriers and
tankers. This leads to missing data and makes it hard to compare ship
types. To keep the data consistent and easy to interpret, TEU and
DWT-based variables were excluded, and gross tonnage (GT), which is
available for all observations, was retained.

\subsubsection{3. Removal of Observations Missing the Response
Variable}\label{removal-of-observations-missing-the-response-variable}

Some country and ship type combinations did not have recorded median
time in port values. Imputing these missing values would have created
artificial outcome data and could have affected the validity of
hypothesis testing. To keep the analysis based only on real observed
port durations, these cases were removed.

\subsubsection{4. Exclusion of Aggregate Categories (``World'' and ``All
ships'')}\label{exclusion-of-aggregate-categories-world-and-all-ships}

The raw dataset contains two broad categories: \textbf{Country =
``World''} and \textbf{ShipType = ``All ships''}. These categories are
more general than the country and ship-type structure used in the
analysis. Including them would combine different units of analysis and
make group comparisons unreliable. To keep the analysis consistent, both
categories were removed.

\subsubsection{5. Log Transformation of the Response
Variable}\label{log-transformation-of-the-response-variable}

To evaluate whether the raw response variable met the distributional
assumptions for linear regression, I examined the distribution of
\textbf{median time in port (days)} using a histogram with density and a
normal Q-Q plot.

\paragraph{Distribution of the Raw Median Time in
Port}\label{distribution-of-the-raw-median-time-in-port}

\begin{center}
\pandocbounded{\includegraphics[keepaspectratio]{plots/Distribution of Median Time in Port.png}}
\end{center}
The histogram of the untransformed response is strongly right-skewed,
with a long upper tail that reaches values of 3 to 4 days or more. Most
observations fall between 0.5 and 1.5 days, but some country and
ship-type combinations have much longer stay durations. This heavy tail
shows that extreme values have a significant impact on the raw data.

\begin{center}
\pandocbounded{\includegraphics[keepaspectratio]{plots/QQ plot for Y.png}}
\end{center}
The Q--Q plot also shows that the data are not normally distributed. The
middle of the distribution matches the theoretical quantiles, but the
upper tail rises sharply above the reference line, showing a clear
departure from normality. This means a transformation might be needed to
stabilize the variance and improve model fits.

\paragraph{Distribution of the Log-Transformed
Response}\label{distribution-of-the-log-transformed-response}

\begin{center}
\pandocbounded{\includegraphics[keepaspectratio]{plots/Distribution of log of Median Time in Port.png}}
\end{center}
A log transformation was applied to the median time in port to address
skewness and reduce the influence of extreme observations. The histogram
of the log-transformed response shows a markedly more symmetric,
bell-shaped distribution. The long right tail is substantially
compressed, and the density appears smoother and more unimodal compared
to the raw scale.

\begin{center}
\pandocbounded{\includegraphics[keepaspectratio]{plots/QQ plot of log of Y.png}}
\end{center}
The Q-Q plot of the transformed variable shows a clear improvement in
normality. Most points are close to the reference line, and there are
fewer deviations at both ends. This means the log transformation helps
reduce the long tails in port-time data.

\paragraph{Summary}\label{summary-1}

Overall, these diagnostics show that the raw response variable is highly
right-skewed and has a long tail. Using a logarithmic transformation
makes the distribution more symmetric, reduces the impact of extreme
values, and matches theoretical normal quantiles more closely. Because
of this, all later regression models use
\(log(\text{median time in port})\) as the response variable.

\section{Methods}\label{sec-methods}

\subsection{Model building}\label{model-building}

Each observation in the data is an aggregated group based on country and
ship type. The modeling aims to find out how vessel characteristics and
other group factors relate to the log-transformed median time in port.

Therefore, we used a hierarchical modeling approach. We fit a series of
models, each adding more parameters to separate the effects of vessel
characteristics, ship type, country differences, and possible
interactions.

\begin{enumerate}
\def\labelenumi{\arabic{enumi}.}
\tightlist
\item
  \textbf{Model 0:} Baseline vessel-structure model
\end{enumerate}

This model examines whether fundamental characteristics of vessels
(average age and average size) are associated with log median time in
port at the group level: \[
\log(\text{Median Time In Port}_i)
= \beta_0
+ \beta_1\,\text{Age}_i
+ \beta_2\,\text{Size}_i
+ \varepsilon_i
\]

\begin{enumerate}
\def\labelenumi{\arabic{enumi}.}
\setcounter{enumi}{1}
\tightlist
\item
  \textbf{Model 1:} Adding ship-type effects
\end{enumerate}

Different ship types are subject to distinct loading, unloading, and
port-handling constraints. To address systematic operational differences
across commercial markets, ship-type indicator variables are
incorporated: \[
\log(\text{Median Time In Port}_i)
= \beta_0
+ \beta_1\,\text{Age}_i
+ \beta_2\,\text{Size}_i
+ \gamma_{j(i)}\,\text{ShipType}_{j(i)}
+ \varepsilon_i
\] This model tests whether, after controlling for age and size, the log
median port time varies among ship types, including dry bulk, liquid
bulk, LNG, and LPG.

\begin{enumerate}
\def\labelenumi{\arabic{enumi}.}
\setcounter{enumi}{2}
\tightlist
\item
  \textbf{Model 2:} Adding country fixed effects
\end{enumerate}

National operational practices influence port performance, and
country-level differences may account for significant variation in port
times. Accordingly, country fixed effects are included: \[
\log(\text{Median Time In Port}_i)
= \beta_0
+ \beta_1\,\text{Age}_i
+ \beta_2\,\text{Size}_i
+ \gamma_{j(i)}\,\text{ShipType}_{j(i)}
+ \delta_{k(i)}\,\text{Country}_{k(i)}
+ \varepsilon_i
\]

\begin{enumerate}
\def\labelenumi{\arabic{enumi}.}
\setcounter{enumi}{3}
\tightlist
\item
  \textbf{Model Int-1:} Size × ShipType interaction.
\end{enumerate}

The effect of vessel size on port time can vary depending on the type of
ship. For example, loading systems for large bulk carriers are different
from those used for large container vessels. To let the effect of size
change with ship type, we add an interaction term: \[
\log(\text{Median Time In Port}_i)
= \beta_0
+ \beta_1\,\text{Age}_i
+ \beta_2\,\text{Size}_i
+ \gamma_{j(i)}\,\text{ShipType}_{j(i)}
+ \theta_{j(i)}\,
    \bigl(\text{Size}_i \times \text{ShipType}_{j(i)}\bigr)
+ \delta_{k(i)}\,\text{Country}_{k(i)}
+ \varepsilon_i
\]

\begin{enumerate}
\def\labelenumi{\arabic{enumi}.}
\setcounter{enumi}{4}
\tightlist
\item
  \textbf{Model Int-2:} Age × ShipType interaction
\end{enumerate}

Similarly, the operational impact of vessel age may differ across ship
types---for example, older bulk carriers might face different delays
than older container ships. Thus, we consider: \[
\log(\text{Median Time In Port}_i)
= \beta_0
+ \beta_1\,\text{Age}_i
+ \beta_2\,\text{Size}_i
+ \gamma_{j(i)}\,\text{ShipType}_{j(i)}
+ \theta_{j(i)}\,
    \bigl(\text{Age}_i \times \text{ShipType}_{j(i)}\bigr)
+ \delta_{k(i)}\,\text{Country}_{k(i)}
+ \varepsilon_i
\]

\subsection{Model Comparison and
Selection}\label{model-comparison-and-selection}

To evaluate competing model specifications, we adopted a sequential
model-building approach, comparing increasingly flexible models using
information criteria and nested hypothesis tests. We began with a
baseline model that included only the numerical vessel characteristics
(Age and Size), and incorporated additional sets of predictors stepwise
to assess their contribution to explaining variation in the
log-transformed median port time.

For each candidate model, we evaluated overall goodness of fit using the
Akaike Information Criterion (AIC) and the adjusted \(R^2\). These
metrics provide complementary assessments of model adequacy and penalize
excessive complexity. Lower AIC values and higher adjusted \(R^2\)
values indicated improved fit.

When models were nested, such as with the addition of ship-type
indicators, country indicators, or interaction terms, we conducted
formal comparisons using likelihood ratio tests. Specifically, we used
F-tests to evaluate whether the additional parameters in the expanded
model significantly improved model fit relative to the simpler
specification. This framework enabled us to determine whether increases
in model complexity were justified by statistically significant
improvements in explanatory power.

We assessed two interaction structures: one between vessel size and ship
type, and another between vessel age and ship type. Each interaction was
evaluated against the main-effects model using the same likelihood-based
criteria. The final model was selected based on parsimony, information
criteria, and statistical evidence from the nested F-tests.

\subsection{Diagnostic Assessment}\label{diagnostic-assessment}

The linear regression model relies on the following key assumptions:

\begin{enumerate}
\def\labelenumi{\arabic{enumi}.}
\tightlist
\item
  \textbf{Linearity}: the relationship between the predictors and the
  response variable is linear.\\
\item
  \textbf{Independence}: each observation is independent of others.\\
\item
  \textbf{Homoscedasticity}: the error term \(\varepsilon_i\) has
  constant variance.\\
\item
  \textbf{Normality}: the errors \(\varepsilon_i\) are normally
  distributed with mean zero.
\end{enumerate}

These assumptions are crucial to the analysis because the data are
grouped by country and ship type. Linearity makes sense here since the
model relates log-transformed port-time durations to average vessel
characteristics, which are summary measures, not individual results.
Normality and homoscedasticity are important for making valid
inferences, such as building confidence intervals and running hypothesis
tests. Independence is reasonable because each observation represents a
different country and ship type, not repeated measures from the same
source.

All diagnostic checks were performed \textbf{only on the final model},
including:

\begin{itemize}
\tightlist
\item
  \textbf{Residual vs.~fitted plot} to check for linearity and equal
  variance.
\item
  \textbf{Normal Q--Q plot} to see if the residuals follow a normal
  distribution.
\item
  \textbf{Scale--location plot} to check if the variance is stable.
\item
  \textbf{Cook's distance} to find any influential data points.
\item
  \textbf{Variance Inflation Factors (VIF / GVIF)} to check for
  multicollinearity.
\end{itemize}

\subsection{Inference and Hypothesis
Testing}\label{inference-and-hypothesis-testing}

Since the main goal of this analysis is inference, not prediction,
formal hypothesis tests were used to evaluate how individual predictors
and groups of related variables contribute. All inference was done using
classical linear regression on the log-transformed median port time.

\subsubsection{t-tests for Individual
Coefficients}\label{t-tests-for-individual-coefficients}

For each regression coefficient in the final model, a two-sided t-test
checked if the predictor is individually associated with the log median
port time, after adjusting for all other terms in the model. For a
coefficient \(\beta_j\), the test evaluates \[
H_0: \beta_j = 0 \quad \text{vs.} \quad H_1: \beta_j \neq 0
\] This shows whether the predictor contributes uniquely to explaining
changes in the response. These tests were used for numerical predictors
like vessel age and size, indicator variables for ship type and country,
and interaction terms between vessel size and ship type.

\subsubsection{F-tests for Groups of
Predictors}\label{f-tests-for-groups-of-predictors}

To see how groups of variables contribute together, such as all
ship-type indicators, all country indicators, or all interaction terms,
nested model comparisons were done using F-tests. For a group of
coefficients \(\{\beta_{j1}, \ldots, \beta_{jk}\}\), the null hypothesis
is
\[H_0: \beta_{j1} = \beta_{j2} = ... = \beta_{jn} = 0 \quad \text{vs.} \quad H_1 : \text{at least one } \beta_{ji}\ \neq 0\]
If \(H_0\) is rejected, it means the group of predictors together
improves the model fit compared to a model without them.

These tests were conducted within the ANOVA by comparing nested models.
For example: - adding ship-type indicators to the baseline numerical
model, - adding country indicators to the ship-type model, - adding size
× ship-type interaction terms to the main-effects model.

F-tests offer a clear way to check if adding more complexity to the
model is justified by significant improvements in how well the model
explains the data.

\subsubsection{Assessment of Interaction
Effects}\label{assessment-of-interaction-effects}

Interaction terms show how vessel size or age affects different ship
types. Their importance was checked using both individual t-tests and a
joint F-test that compares the interaction model to the main-effects
model. This way, the analysis considers both individual and group
effects of the interactions.

\subsubsection{Inference Scope}\label{inference-scope}

All hypothesis tests were done only on the final model chosen through
model comparison. The t-tests and F-tests together help show which
vessel features, market segments, and country-level factors are
associated with differences in log median port time at the country and
ship-type level.

\section{Results}\label{sec-results}

\subsection{Final Model Comparison
Table}\label{final-model-comparison-table}

\begin{center}
\pandocbounded{\includegraphics[keepaspectratio]{Tables/Model Comparison Table.png}}
\end{center}
Model Int-1 was selected as the final model because it hachieved the
lowest AIC, the highest adjusted \(R^2\), and a statistically
significant improvement over the main-effects model (Model 2) based on a
nested F-test \((p = 0.0125)\). The alternative interaction model (Model
Int-2) did not significantly improve model fit \((p = 0.213)\) and had a
higher AIC.

\subsection{Final Model(Model Int-1)}\label{final-modelmodel-int-1}

\[
\log(\text{Median Time In Port}_i)
= \beta_0
+ \beta_1\,\text{Age}_i
+ \beta_2\,\text{Size}_i
+ \gamma_{j(i)}\,\text{ShipType}_{j(i)}
+ \theta_{j(i)}\,
    \bigl(\text{Size}_i \times \text{ShipType}_{j(i)}\bigr)
+ \delta_{k(i)}\,\text{Country}_{k(i)}
+ \varepsilon_i
\]

\subsubsection{Estimate Coefficent}\label{estimate-coefficent}

Table Table~\ref{tbl-model-int1} shows the estimated regression
coefficients from the final model. Since the data are grouped by country
and ship type, each coefficient describes differences between these
groups, not individual vessels or ports. The response variable is the
log of median port time, so all effects reflect percentage changes in
median port time.

When vessel size, ship type, country, and the size-by-ship-type
interaction are considered, the estimated effect of average vessel age
is small and not statistically significant
\((\hat{\beta}_1 = -0.0054,\; p = 0.69)\). This means that differences
in average vessel age between country and ship-type groups do not show
clear evidence of systematic differences in median port time on the log
scale.

On the other hand, average vessel size has a positive and statistically
significant association with log median port time
\((\hat{\beta}_2 = 1.19 \times 10^{-5},\; p = 0.039)\). Holding the
other factors constant, this estimate indicates that, across aggregated
country--ship-type groups, units with larger average vessel sizes tend
to have longer median port times. Since the response is log-transformed,
the coefficient shows a proportional effect: a \(10,000-GT\) increase in
average vessel size is associated with an approximate 0.12\% increase in
median port time, all else being equal.

Some commercial market segments statistically significant differences
compared to the reference ship type. For example, Dry Bulk carriers
\((p = 0.034)\) and LNG carriers \((p = 0.037)\) have higher log median
port times after adjusting for vessel size, country, and the
size-by-ship-type interaction. With all other terms held constant, these
coefficients represent group-level shifts in log port time, suggesting
that features specific to these market segments may lead to longer port
times.

The interaction terms between vessel size and ship type are mostly not
statistically significant at the 5\% level, except for LNG carriers,
which is marginally significant \((p = 0.056)\). This gives limited
evidence that the link between vessel size and median port time changes
across market segments. The marginal result for LNG carriers may suggest
a different size effect, but the evidence is not strong enough to be
conclusive.

Country indicators also show different baseline levels of log median
port time. For example, Canada \((p = 0.0039)\) and Italy
\((p = 0.033)\) have port-time intercepts that are significantly
different from the reference country. These coefficients should be seen
as additive shifts on the log scale, not as causal effects, and they
reflect structural differences like port efficiency, regulations, or
measurement differences in the grouped data.

Many ship-type and country indicators are not statistically significant
on their own, which is expected in models with many categories and a
limited sample size. At the grouped country--ship-type level, some
categories may not vary enough to show clear differences. So, if a
coefficient is not significant, it means there is no statistical
evidence for a difference, not that the operational characteristics are
the same across categories.

\subsubsection{Residual Standard Error \& Adjusted
R-Squared}\label{residual-standard-error-adjusted-r-squared}

The residual standard error of 0.394 indicates that, on the log scale,
the typical difference between observed and fitted median port times is
moderate compared to the overall variability in the data. Since the
dataset is grouped by country and ship type, this level of residual
variability is expected. There is significant operational diversity
within each group that the available predictors cannot fully explain.

The adjusted \(R^2\) of 0.51 means the model explains about half of the
variation in log median port time. This level of explanatory power is
typical for aggregated operational data in maritime logistics, where
many unobserved factors affect port times. Since this study focuses on
inference rather than prediction, the adjusted \(R^2\) serves mainly as
a descriptive measure of how well the model fits, not as a basis for
choosing the model.

In summary, the reasonable residual error and moderate adjusted \(R^2\)
show that the final model is suitable for estimating how vessel
characteristics and market segments relate to median port times at the
country and ship-type level.

\subsubsection{ANOVA}\label{anova}

\paragraph{ANOVA Table for Model Int1}\label{tab-anova}

\begin{figure}[H]

{\centering \pandocbounded{\includegraphics[keepaspectratio]{Tables/ANOVA.png}}

}

\caption{Model Comparison}

\end{figure}%

The ANOVA results for the selected interaction model show that several
groups of predictors significantly help explain variation in log median
port time across the combined country and ship type observations.

Average vessel age does not explain much of the variation in the
response \((F = 0.59,\; p = 0.44)\), which matches earlier findings that
there is little evidence that age differences across groups are
associated with differences in port-time performance.

On the other hand, average vessel size is a significant factor in the
model \((F = 9.61,\; p = 0.0026)\). This means that differences in
vessel size across country and ship type groups are related to
consistent differences in median port time, even after controlling for
other predictors and interaction effects.

The Ship Type shows a highly significant effect
\((F = 12.14,\; p < 4.6\times10^{-9})\). This shows that the way
different vessel types operate is closely associated with differences in
port-time outcomes at the aggregated level.

Country indicators also make a statistically significant contribution
\((F = 3.62,\; p < 9.7\times10^{-6})\). This means that even after
considering vessel characteristics and market segment, there are still
important differences between countries, likely due to institutional,
regulatory, or infrastructure factors.

Finally, the interaction between vessel size and ship type is
statistically significant \((F = 3.10,\; p = 0.0125)\). This shows that
the effect of vessel size on median port time changes across different
ship type. While some individual interaction coefficients may not seem
significant in the regression summary, the joint F-test shows that,
together, these interaction terms make the model fit better.

Overall, these ANOVA results show that vessel size, ship-type market
segment, country-level characteristics, and the interaction between size
and ship type all play important roles in explaining variation in log
median port time at the aggregated country--ship-type level.

\subsubsection{Evaluation of
Assumptions}\label{evaluation-of-assumptions}

\begin{center}
\pandocbounded{\includegraphics[keepaspectratio]{plots/Residuals vs Fitted.png}}
\end{center}
- \textbf{Residuals vs.~Fitted Values}: The residuals do not show clear
patterns, curvature, or funnel shapes. The points are spread fairly
evenly around zero for all fitted values, which means the linearity and
constant error variance assumptions hold. Some moderate scatter is
normal because the country and ship-type groups are different, but the
plot does not show any major problems.

\begin{center}
\pandocbounded{\includegraphics[keepaspectratio]{plots/Normal Q-Q Plot.png}}
\end{center}
- \textbf{Normal Q--Q Plot}:\\
The Q-Q plot indicates that most residuals form a nearly straight line
along the reference line, with only slight deviations at the ends. These
small differences do not suggest serious non-normality.

\begin{center}
\pandocbounded{\includegraphics[keepaspectratio]{plots/Scale-Location.png}}
\end{center}
- \textbf{Scale--Location Plot}:\\
The plot of the square root of standardized residuals versus fitted
values shows a fairly even spread across the fitted range, with no clear
signs of heteroscedasticity. Although there is some variability, it does
not seem to increase or decrease as fitted values change. This suggests
that the constant variance assumption for the log-transformed response
is reasonable.

\begin{center}
\pandocbounded{\includegraphics[keepaspectratio]{plots/Cook's Distance.png}}
\end{center}
- \textbf{Cook's Distance}:\\
Most observations have very small Cook's distance values, which means
they have little effect on the fitted model. Only a few points go above
the usual reference threshold, showing that some country and ship-type
groups are more influential. Again, these points are not extreme enough
to make the coefficient estimates unstable or to justify removing them.
Their influence shows real structural differences in the data, not
problems with data quality.

\begin{figure}[H]

{\centering \pandocbounded{\includegraphics[keepaspectratio]{Tables/GVIF.png}}

}

\caption{Variance Inflation Factors (VIF / GVIF)}

\end{figure}%

\begin{itemize}
\tightlist
\item
  \textbf{Multicollinearity Assessment}: Generalized variance inflation
  factors (GVIFs) were calculated to check for multicollinearity, since
  the model uses categorical predictors with multiple degrees of freedom
  and interaction terms. The adjusted measure,
  \(\text{GVIF}^{1/(2 \cdot \text{Df})}\), is similar to a standard VIF.
\end{itemize}

All adjusted GVIF values are between 1 and 5, so there is no strong
multicollinearity. Vessel size has a moderate adjusted VIF \((4.86)\),
which is expected because it appears in both the main effect and the
size by ship-type interaction. The categorical predictors, ship type and
country, have low adjusted GVIF values even though their raw GVIF values
are high. This is due to their multi-level structure, not to
collinearity problems.

Overall, multicollinearity is not a problem for coefficient stability or
the validity of inferences in the final model.

\section{Discussion}\label{sec-discussion}

\section{Reproducibility}\label{reproducibility}

\section{References}\label{references}

\section{Appendix}\label{sec-appendix}

\begin{table}

\caption{\label{tbl-model-int1}Table: Estimated coefficients from
Model\_int1}

\centering{

\begin{figure}
\centering
\pandocbounded{\includegraphics[keepaspectratio]{Tables/Model-int1_summary.png}}
\caption{Estimation of Model-int1}
\end{figure}

}

\end{table}%




\end{document}
